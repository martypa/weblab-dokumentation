\documentclass[../main.tex]{subfiles}
\graphicspath{{images/}{../images/}}

\begin{document}
Das Projekt Travelblog konnte in dem Umfang realisiert werden wie geplant. Folgende Funktionalitäten konnten erfolgreich implementiert werden:
\begin{itemize}
    \item Anzeigen von mehreren Blogreisen
    \item Anzeigen von Blogeinträgen pro Reise
    \item Erstellen von neuen Reisen mittels Formular
    \item Erstellen von neuen Blogeinträgen mit Bildupload
    \item Löschen von Blogeinträgen und Reisen
    \item Userbasierende Authorisierung der obigen Funktionen
\end{itemize}

Das Travelblog Projekt war für mich das erste Fullstack Webprojekt. Beruflich komme ich aus dem Backendbereich. Es war aber interessant auch mal ein Frontendprojekt zu realisieren. Die Entscheidung auf bereits etablierte Technologien wie Angluar8 und Nodejs zu setzten, hat sich im nachhinein für sehr wertvoll bewiesen. Bei Fragen oder Unklarheiten konnte das Internet mit vielen stackoverflow Einträgen oder MediumBlogs mit Rat zur Seite stehen. Der Aufwand für ein Fullstack Projekt ist sehr gross, vorallem wenn man dies alleine druchführen will. Ich habe die IDE von Jetbrains Webstorm verwendet. Dadurch konnte ich ohne Probleme Fehlersuche mittels Debugging betreiben. Ich habe von anderen Mitstudierenden erfahren, dass dies bei VisualCode eher mühsam war. Alles in allem war es eine tolle Blockwoche und ein interessantes Projekt, welches aber viel Zeit in anspruch nahm.
\end{document}