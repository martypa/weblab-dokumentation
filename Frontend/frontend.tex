\documentclass[../main.tex]{subfiles}
\graphicspath{{images/}{../images/}}

\begin{document}
Als Clientseites Frontendframework wurde im Travelblog Projekt Angular8 eingesetzt. Dies aus dem Grund, dass es eine etablierte, breit eingesetztes mit grosser Community versehens Framework ist. Für eine saubere und zeitgemässe Darstellung wurde die Angular Material Library verwendet. Mit Angular Material können einfach vordefinierte Komponenten im Material Design stil erstellt werden.\\
\subsection{Aufbau}
Der Aufbau der Applikation ist sehr einfach gehalten. Folgende Komponenten kommen bei der Applikation zum Einsatz:
\begin{itemize}
    \item dashboard: Einstiegspunkt der Applikation mit der tabellenförmigen Darstellung aller erfassten Reisen
    \item travel-blog-entry: Hier werden alle erstellten Blogeinträge der ausgewählten Reise dargestellt. 
    \item blog-editing: Hier können neue Blogeinträge für eine Reise erstellt werden.
    \item create-travel: Hier kann eine neue Reise erstellt werden.
    \item authentication-login: Hier können sich die Benutzer mit Benutzername und Passwort anmelden.
\end{itemize}

\subsection{Blogeinträge / Reisen erstellen}
Es können mittels Formulare Reisen oder Blogeinträge erstllt werden. Bei den Blogeinträgen kann man auch optional ein Bild hinzugefügt werden. Die Funktionalität Bildupload wird mit der 3rd Party Library ng2-file-upload realisiert. Die Blogeinträge respektive die Reiseinformationen werden als JSON Objekt mittels POST-Methode dem Server zur Persisitierung übergeben.

\subsection{Authentifizierung / Authorisierung}
Die Authentifizierung wird mittels Benutzername und Passwort Eingabe durchgeführt. Das Passwort wird direkt in einen SHA-256 Hash umgewandelt, damit dieses sicher zum Server übermittelt werden kann. Bei erfolgreicher Authentifizierung durch den Server wird der Benutzername, seine Rolle und das Token an die Applikation zurückgegeben. Mittels der Rolle (creator, visitor) kann der Benutzer nun auf unterschiedliche Funktionen zugreiffen.\\

\textbf{Creator:}\\
- Reisen erstellen und löschen \\
- Blogeinträge lesen und löschen\\

\textbf{Visitor:}\\
- Reisen und Blogeinträge lesen\\

\end{document}